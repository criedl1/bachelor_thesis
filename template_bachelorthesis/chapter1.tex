This document should serve as template for your thesis and as a quick-start for {\LaTeX}.  In this document we give a generic {\em example} of a thesis organization and we demonstrate the usage of some basic commands typically required for writing a scientific report.  This document is neither complete nor does it represent the ''only'' possible solution for a scientific report.  There are many sources for more information about {\LaTeX} \cite{weinelt} or writing a scientific report \cite{Day2006}.
Note that we {\bf do not require} you to use {\LaTeX} for writing your thesis, but we strongly recommend it. 

\section*{Organization of this template}
This template consists of the following files:
\begin{itemize}
\item{\tt masterthesis.tex} Main file (Latex) 
\item{\tt titlepage.tex} Title page (Latex)
\item{\tt stmt.tex} Text for the statutory declaration (Latex)
\item{\tt abstract\_e.tex} Text for the English abstract (Latex)
\item{\tt abstract\_g.tex} Text for the German abstract (Latex)
\item{\tt acknow.tex} Text for the acknowledgements (Latex)
\item{\tt chapter1.tex} Text for chapter 1 (Latex)
\item{\tt chapter2.tex} Text for chapter 2 (Latex)
\item{\tt chapter3.tex} Text for chapter 3 (Latex)
\item{\tt KLU-logo.eps} Logo of Klagenfurt University (image)
\item{\tt fig.eps} Sample figure (image)
\item{\tt reference.bib} References for this text (Bibtex)
\item{\tt masterthesis.pdf} Final output (PDF)
\end{itemize}

\newpage
\noindent
The remainder of this document consists of dummy text to demonstrate frequently used commands.

A chapter can be organized in several sections, subsections, subsubsections and paragraphs.

\section{Motivation}

\subsection{A Subsection}

\subsubsection{A Subsubsection}

This is the text which follows a heading.

A new paragraph can be simply started by inserting an empty line.  Note that in the standard documentstyle all but the first paragraphs are indented. 

\section{Goals}

\section{Outline of the Thesis}

