
Chapter~\ref{sec:sota} gives a brief overview of the state of the art
of this research.

\section{A Sample Text}
 
Over the years, human experts have gained sufficient experience to 
solve a lot of problems
efficiently. During the last years, much work has been done to build
machines that imitate the description of human thinking and acting.
Expert systems are computer programs which use knowledge and inference
mechanisms to solve problems where normally expert knowledge is required.
One important class of problems is {\em diagnostic reasoning}.
Diagnostic reasoning can be seen as a classification problem, since it involves
identifying current behavior with a set of known classes of behavior. It can 
also be considered as an abduction problem, concerned with generating plausible
explanations for observations \cite{Kui86}.

Diagnostic reasoning systems are important in many technical systems.
As these systems---like electronic circuits, assembly 
lines or nuclear power plants---are becoming more complex, the need for 
automatic reasoning systems to support troubleshooting is increasing 
enormously.

Most diagnostic expert systems can be classified into the following categories:

\begin{description}
\item[Statistical approach.]
The statistical approach determines the probabilities of a diagnosis. This 
approach is mainly based on the {\em Bayes Theorem\/}. Due to the lack of 
causal inference, the statistical reasoning process has only a very limited 
ability for explanation.

\item[Associative approach.]
Associative diagnosis systems are built by accumulating the experience of
expert troubleshooters in the form of empirical associations. The associative
approach became popular, in part, because it permitted easy construction of 
expert systems by encoding heuristic information in the form of 
{\em if--then\/} rules. However, a big problem is the knowledge accumulation
of rule-based systems.

\item[Model-based approach.]
Model-based diagnosis can be viewed as an interaction between prediction and
observation. The knowledge is represented by different models for the 
{\em structure\/} and the {\em behavior\/} of the system. The predicted 
behavior is compared with the actual observation, producing  
discrepancies. Discrepancies then give rise to a possible diagnosis.
A Model-based diagnosis covers a broader range of faults by viewing misbehavior
as anything other than what the model predicts. This approach also better
captures the causal dependencies of the system than the other two approaches.
\end{description}

The key units of a model-based diagnostic expert system are the {\em models\/} 
which describe behavior and structure and the {\em inference 
mechanism\/} for predicting the behavior given the structure of the system. 
Reasoning with models can be broken down into two major subproblems 
\cite{Kui86}.
{\em Model building\/} starts with a description of the physical situation and 
builds an appropriate simplified model. {\em Model simulation\/} starts with
a model and predicts the possible behaviors consistent with the model.
A system can be modeled at different levels of abstractions. These levels
range from detailed numerical descriptions up to rather coarse and incomplete
descriptions at a qualitative level. The selected abstraction level depends
basically on the domain, the information available about the system and the
inference mechanism which must correspond to the model description. 
In many cases, the behavior is predicted by simulators. Therefore, 
numerical, discrete or qualitative simulators are often applied. 
Sometimes system descriptions at different levels are combined to achieve 
diagnosis at an appropriate level of detail.

\subsection{A few References}

Model-based fault diagnosis is applied more and more nowadays. This reasoning
technique is used in both static as well as dynamic systems. In dynamic 
systems, the parameters of the system change over time. Hence, the behavior 
also changes over time. Examples for model-based diagnosis in static systems
are \cite{Bramberger_Computer2006, Bramberger_RTAS2004}. Model-based diagnosis in dynamic systems is 
demonstrated in \cite{Patterson2006} \cite{Aghajan_ICDSC2007}.

